%\documentclass[russian,english,10pt,a5paper,reqno]{amsart}
%\documentclass[russian,english,10pt,a4paper,reqno]{article}
%\documentclass[russian,english,11pt,b5paper,reqno,dviphfm]{amsbook}
%\documentclass[russian,english,10pt,a5paper,reqno,dviphfm]{amsbook}
%\documentclass[russian,english,10pt,a5paper,reqno,dviphfm]{amsbook}
\documentclass[russian,english,18pt,a4paper,reqno,dviphfm]{article}


%%\usepackage[pdftex,a5paper]{hyperref}
\usepackage[headings]{fullpage}
%\usepackage{fancyhdr}
%%\setlength{\textwidth}{114mm}
%%\setlength{\linewidth}{114mm}
%%\setlength{\textheight}{175mm}
% последние три команды позволяют настроить размеры текста на странице
%%\hoffset=-8mm
%%\voffset=-13mm
%% предшествующие две команды позволяют избавиться от отступов по-умолчанию
%%\setlength\paperheight{210mm}
%%\setlength\paperwidth{148mm}
%% Последние две команды пришлось написать в явном виде,
%% поскольку ничего другое pdflatex не понимал

\setlength{\footskip}{7mm}
\usepackage[T2A]{fontenc}
\usepackage[utf8]{inputenc}
\usepackage[russian]{babel}
\usepackage{amsmath}
\usepackage{amssymb}
\usepackage{amsfonts}
\usepackage{textcomp}
\usepackage[all]{xy}
\usepackage{amsthm}
\usepackage{graphicx}
%\usepackage[dvips]{graphicx}
\usepackage{wrapfig}
\usepackage{concrete}
\usepackage{eufrak}
%\usepackage{euler}
\usepackage{babelbib}
\usepackage{multirow}
\usepackage{multicol}
\usepackage{longtable}
\usepackage{cite}
\usepackage{ifthen}
\usepackage{array}
\usepackage{soul}
\usepackage{indentfirst}
\usepackage{varioref}
\usepackage{hyperref}
\usepackage{enumitem}
% Default fixed font does not support bold face
\DeclareFixedFont{\ttb}{T1}{txtt}{bx}{n}{8} % for bold
\DeclareFixedFont{\ttm}{T1}{txtt}{m}{n}{8}  % for normal

% Custom colors
\usepackage{color}
\definecolor{deepblue}{rgb}{0,0,0.5}
\definecolor{deepred}{rgb}{0.6,0,0}
\definecolor{deepgreen}{rgb}{0,0.5,0}

\usepackage{listings}

% Python style for highlighting
\newcommand\pythonstyle{\lstset{
language=Python,
basicstyle=\ttm,
otherkeywords={self},             % Add keywords here
keywordstyle=\ttb\color{deepblue},
emph={MyClass,__init__},          % Custom highlighting
emphstyle=\ttb\color{deepred},    % Custom highlighting style
stringstyle=\color{deepgreen},
frame=tb,                         % Any extra options here
showstringspaces=false            % 
}}


% Python environment
\lstnewenvironment{python}[1][]
{
\pythonstyle
\lstset{#1}
}
{}

% Python for external files
\newcommand\pythonexternal[2][]{{
\pythonstyle
\lstinputlisting[#1]{#2}}}

% Python for inline
\newcommand\pythoninline[1]{{\pythonstyle\lstinline!#1!}}
\begin{document}

\pdfoutput=1

\selectlanguage{russian}
\begin{titlepage}
  \begin{center}

    САНКТ-ПЕТЕРБУРГСКИЙ ГОСУДАРСТВЕННЫЙ УНИВЕРСИТЕТ
    \vspace{3.25cm}

    Математико-механический факультет

    Кафедра информатики
    \vspace{3.25cm}


    Лысов Александр Васильевич \\
    Епрев Артем Евгеньевич
    \vfill

    \textsc{Курсовая работа}\\[5mm]

    {\LARGE{Исследование работы батарей с применением LSTM~RNN}}
    \bigskip

    3 курс, группа 342
  \end{center}
  \vfill

  \newlength{\ML}
  \settowidth{\ML}{«\underline{\hspace{0.7cm}}» \underline{\hspace{2cm}}}
  \vfill
  \hfill\begin{minipage}{0.4\textwidth}
    Руководитель курсовой работы\\
    \underline{\hspace{\ML}} А.\,А.~Алиев\\
    «\underline{\hspace{0.7cm}}» \underline{\hspace{2cm}} 2017 г.
  \end{minipage}%

  \vfill

  \begin{center}
    Санкт-Петербург, 2017 г.
  \end{center}
\end{titlepage}

\newpage

\tableofcontents

\newpage

\section{Введение}
Целью нашей работы являлось исследование работы батарей. \\
\indentЛитий-ионные батареи используется повсюду: от наручных часов до электрокаров, поэтому их ислледование является необходимым для реального мира. \\
\indentВ данной курсовой работе будет исследована зависимость емкости батареи от того, как проходит ее разряд.
\newpage
\section{Основная часть}
\subsection{Начальный набор данных}
Были взяты наборы данных циклов зарядов-разрядов батарей.
		Набор данных был взят с сайта ti.arc.nasa.gov\footnote{https://ti.arc.nasa.gov/tech/dash/pcoe/prognostic-data-repository/\#battery} и был представлен в расширении .mat 
		\\
		\indentНабор из четырех литий-ионных батарей (№ 5, 6, 7 и 18) выполнялся через 3 различных рабочих профиля (заряд и разряд) при комнатной температуре. Зарядка проводилась в режиме постоянного тока (CC) при 1,5 А до тех пор, пока напряжение батареи не достигло 4,2 В, а затем продолжалось в режиме постоянного напряжения (CV), пока зарядный ток не упал до 20 мА. 
		\\
		\indentРазряд проводился при постоянном токе (CC), равном 2А, до тех пор, пока напряжение батареи не упало до 2,7 В, 2,5 В, 2,2 В и 2,5 В для батарей 5 6 7 и 18 соответственно. 
		\\
		\indentЭксперименты были прекращены, когда батареи достигли критериев окончания срока службы (EOL), что на 30\% снизилось в номинальной емкости (от 2Ahr до 1.4Ahr).
		В датасете были следующие параметры для соответствующих циклов: 
		\begin{itemize}
			\item[charge:]
				\begin{tabular}{ |l|l| }
				  \hline
						Voltage\_measured & Battery terminal voltage (Volts) \\\hline
						Current\_measured & Battery output current (Amps) \\\hline
						Temperature\_measured & Battery temperature (degree C) \\\hline
						Current\_charge & Current measured at charger (Amps) \\\hline
						Voltage\_charge & Voltage measured at charger (Volts) \\\hline
						Time & Time vector for the cycle (secs) \\
				  \hline
				\end{tabular}
			\item[discharge:]
			\begin{tabular}{ |l|l| }
			  \hline
					Voltage\_measured & Battery terminal voltage (Volts) \\\hline
					Current\_measured & Battery output current (Amps) \\\hline
					Temperature\_measured & Battery temperature (degree C) \\\hline
					Current\_charge & Current measured at load (Amps) \\\hline
					Voltage\_charge & Voltage measured at load (Volts) \\\hline
					Time & Time vector for the cycle (secs) \\\hline
					Capacity & Battery capacity (Ahr) for discharge till 2.7V \\
			  \hline
			\end{tabular}
		\end{itemize}
\newpage
\subsection{Обработка данных}
После загрузки набора данных в расширении .mat была получена неудобная для дальнейшего использования структура. Было решено переформатировать ее в расширение .csv со структурой следующего вида:\\
\begin{tabular}{ccccc}
\multicolumn{5}{c}{x\_dataset}\\
label1\_1;&label1\_2;&label1\_3;&$\cdots$;&label1\_N\\
label2\_1;&label2\_2;&label2\_3;&$\cdots$;&label2\_N\\
$\vdots$&&&$\ddots$&\\
labelM\_1;&labelM\_2;&labelM\_3;&$\cdots$;&labelM\_N
\end{tabular} и 
\begin{tabular}{c}
y\_dataset\\
res1\\
res2\\
$\vdots$\\
res3\\
\end{tabular} \\\\
Реализация тренировочного и тестового набора данных признаков:
\begin{python}
fileName="B0006"
pathDS="ds/"
a = loadmat("d1/"+fileName+".mat")
newDS=open(pathDS+"xd_train.csv", "w")
for i in range(616):
    numbOfFeat=len(a[fileName][0, 0][0][0][i][3][0][0])
    numbOfVect=len(a[fileName][0, 0][0][0][i][3][0][0][0][0])    
    typeCycle=a[fileName][0, 0][0][0][i][0][0]
    if (not typeCycle=="discharge"):
        continue
    ar=np.linspace(0, numbOfVect-1, 10, dtype=int)
    for j in ar:
        for k in range (6): 
            newDS.write(str(a[fileName][0, 0][0][0][i][3][0][0][k][0][j]))
            if (not (j==ar[-1] and k==5)):
                newDS.write(";")
    newDS.write("\n")
newDS.close()
\end{python}

\noindentРеализация тренировочного и тестового набора целевых переменных:
\begin{python}
pathDS="ds/"
fileName="B0006"
xS=1
a = loadmat("d1/"+fileName+".mat")
newDS=open(pathDS+"y_train.csv", "w")
for i in range(616):
    numbOfFeat=len(a[fileName][0, 0][0][0][i][3][0][0])
    typeCycle=a[fileName][0, 0][0][0][i][0][0]
    if (not typeCycle=="discharge"):
        continue
    for h in range(xS):
        newDS.write(str(a[fileName][0, 0][0][0][i][3][0][0][6][0][0])+"\n")
newDS.close()
\end{python} 

\newpage
\subsection{Идеи методов}
Предварительно исследовав изменения значения параметра емкости батареи на протяжении всех циклов (изменение было как минимум не линейным)\ref{fig:1}, а также учитывая большое количество признаков в каждом из объектов, было решено использовать нейронные сети. \\
\indentПостроение нейронных сетей явялется одним из самых передовых и хорошо зарекомендовавших себя подходов в машинном обучении. Так как для адекватного предсказания требовалось учитывать структуру данных (временной ряд), было решено использовать рекуррентную нейронную сеть с долгой краткосрочной памятью (recurrent neural network long short-term memory)\ref{fig:2}. \\
Наш метод решения данной проблемы состоит из нескольких этапов:
\begin{enumerate}
	\item Поиск наиболее подходящего набора данных (датасета);
	\item Обработка датасета, перевод из структуры MatLab (.mat) в привычный .csv;
	\item Построение графиков зависимости емкости батареи от параметров циклов разрядки;
	\item Перевод каждого цикла разрядки, в котором содержалось различное количество векторов состояний, в один вектор, понятный для алгоритма;
	\item Построение LTSM рекуррентной нейросети;
	\item Отбор наилучших параметров алгоритма;
	\item Проверка точности на тестовых данных
\end{enumerate}
\begin{figure}[h!]
	\center
	\includegraphics[width=0.4\textwidth]{pic/cap.png}
	\caption{График зависимости capacity от номера цикла разряда} 
	\label{fig:1}
\end{figure}
\begin{figure}[h!]
	\center
	\includegraphics[width=0.7\textwidth]{pic/LSTM3-chain.png}
	\caption{Принцип LSTM рекуррентной нейронной сети} 
	\label{fig:2}
\end{figure}
\newpage
\subsection{Реализация метода}
Программа написана на языке $Python 2$, и в ней использовались библиотеки $keras$, $numpy$, $matplotlib$, $pandas$, $sklearn$ и $scipy$. \\\\
Чуть подробнее о каждой из библиотек:
\begin{itemize}
	\item[$Keras$] Библиотека глубинного обучения с достаточным количеством метрик, оптимизаторов и функций потерь, которая может применяться на основе $Theano$ или $TensorFlow$. Было решено использовать backend $Theano$, чтобы попробовать распараллелить на CUDA, но, к сожалению, имеющаяся видеокарта NVidia GeForce 610M оказалась слишком старой и значительного прироста это не дало, только ускорение в 1.5---2 раза;
	\item[$NumPy$] Библиотека, из которой были использовали numpy-массивы для ускорения работы реализации программы;
	\item[$Matplotlib$] Библиотека для построения графиков;
	\item[$Pandas$] Была использована для загрузки .csv датасетов;
	\item[$Sklearn$] Еще одна библиотека, используемая для машинного обучения. Для качества измерения использовали различные метрики из нее;
	\item[$SciPy$] Была нужна для загрузки изначального .mat набора данных.
\end{itemize}
Процесс реализации состоял из следующих этапов:
\begin{enumerate}
	\item При помощи метода loadmat из scipy.io был загружен изначальный набор данных циклов батарей. Структура была неудобна для дальнейшего использования, а именно имела вид: 
	\begin{python}
dataset[fileName][0, 0][0][0][i][3][0][0][k][0][j]
	\end{python},
	где $i$ --- номер цикла, $k$ --- номер вектора в цикле, $j$ --- номер параметра в векторе.
	\item Выбор тренировочного и тестового датасетов. Так как батареи $6$ и $18$ разряжались до одинакового напряжения (2.5 В), было решено выбрать батарею $6$ в качестве тренировочного датасета, а $18$ в качестве тестового датасета.
	\begin{itemize}
		\item Создание тренировочного и тестового датасетов признаков: \\ 
		Поскольку один цикл разрядки представлял собой матрицу $Nx6$, где $N$ был от $180$ до $371$, было решено брать 10 равнораспределенных векторов при помощи: \begin{python}
np.linspace(0, numbOfVect-1, 10, dtype=int)
		\end{python} и объединять их в один для каждого цикла. 
		В итоге у нас получилось два набора данных: x\_train ($168x60$) и x\_test ($132x60$);
		\item Создание тренировочного и тестового датасетов целевой переменной. \\ 
		В каждом цикле разрядки также содержалось одно значение емкости батареи (capacity). Целью построенной рекуррентной нейронной сети являлось предсказание данного значения по входным векторам состояния цикла разрядки батареи. Были также созданы 2 набора данных целевой переменной: y\_train ($168x1$) и y\_test ($132x1$);
	\end{itemize}
	В ходе построения LSTM рекуррентной нейросети были использованы слои: 
	\begin{itemize}
	    \item[] keras.layers.Dense(60, input\_shape=(60,))
	    \item[] keras.layers.Reshape((60,1)))
	    \item[] keras.layers.LSTM(60, return\_sequences=True)
	    \item[] keras.layers.LSTM(60, dropout=0.3, recurrent\_dropout=0.32)
	    \item[] keras.layers.Dense(1)
	\end{itemize}
	В качестве функции потерь (loss) была выбрана среднеквадратическая ошибка, в качестве оптимизатора (optimizer) был выбран  adam, поскольку эти параметры оказалось наиболее подходящими для решения поставленной задачи и при помощи них была подсказана наиболее точная модель поведения capacity с mean\_sqrt=$0.000784$. 
	\item Проверка метрикой mean\_square\_error для наших результатов и тестового набора результатов, который был равен $0.000784$
\end{enumerate}

\newpage

\section{Заключение}
В итоге на тестовых данных обученная рекуррентная нейронная сеть построила модель\ref{fig:3} со среднеквадратической ошибкой $0.000784$
\begin{figure}[h!]
	\center
	\includegraphics[width=0.5\textwidth]{pic/model.png}
	\caption{Построенная модель. Желтым --- тестовые данные, синим --- результат построенной рекуррентной нейросети} 
	\label{fig:3}
\end{figure}

Дальнейшие цели:
\begin{enumerate}
 	\item 1 
 	\item 2
\end{enumerate}
\newpage
\pagestyle{plain}
\section*{Список литературы}
\begin{enumerate}
	\item \href{https://en.wikipedia.org/wiki/Long\_short-term\_memory}{https://en.wikipedia.org/wiki/Long\_short-term\_memory}
	\item \href{http://machinelearningmastery.com/time-series-prediction-lstm-recurrent-neural-networks-python-keras/}{http://machinelearningmastery.com/time-series-prediction-lstm-recurrent-neural-networks-python-keras/}
	\item \href{https://keras.io/metrics/}{https://keras.io/metrics/}
	\item \href{https://keras.io/losses/}{https://keras.io/losses/}
	\item \href{https://keras.io/optimizers/}{https://keras.io/optimizers/}
	\item \href{https://keras.io/models/model/}{https://keras.io/models/model/}
	\item \href{https://ti.arc.nasa.gov/tech/dash/pcoe/prognostic-data-repository/\#battery}{https://ti.arc.nasa.gov/tech/dash/pcoe/prognostic-data-repository/\#battery}
\end{enumerate}
\end{document}
\bye
